\section{Attributes [C++]}
\label{sec:attributes}

\experimental

\lang allows decorating declarations with \textit{attributes}.
These can be thought of as supplying small amounts of specification information.

\subsection{[[noreturn]]}
\label{sec:noreturn}

The \lstinline|[[noreturn]]| attribute on a C++ function declaration means
that the function never returns: if it terminates at all, it always throws
exceptions or exits abruptly. Consequently, the only appropriate \lstinline|ensures| clause
is \lstinline|ensures \\false;|.

Thus
\begin{itemize}
\item It is a syntactic error if a function has both a \lstinline|[[noreturn]]|
attribute and an \lstinline|ensures| clause whose predicate is not the
boolean literal \lstinline|\\false|.
\item If a function with an \lstinline|[[noreturn]]| attribute has a behavior
that contains no \lstinline|ensures| clause, the default \lstinline|ensures|
clause is \lstinline|ensures \\false;| (rather than the usual \lstinline|ensures \true;|).
\end{itemize}

\begin{example} The following code demonstrates this feature:
\begin{listing-nonumber}
/*@
ensures \false;
ensures \true; // ILLEGAL in combination with [[noreturn]]
*/
void m() [[noreturn]] { ... }
\end{listing-nonumber}
\end{example}

\ifImpl{Status: [[noreturn]] is noted and the implications for specifications are checked.}

\subsection{noexcept}
\label{sec:noexcept}

The \lstinline|[[noexcept]]| attribute on a C++ function declaration means
that the function never throws an exception: if it terminates at all, it always terminates normally. Consequently, the only appropriate \lstinline|throws| clause
is \lstinline|throws \\false;|.

Thus
\begin{itemize}
	\item It is a syntactic error if a function has both a \lstinline|[[noexcept]]|
	attribute and a \lstinline|throws| clause whose predicate is not the
	boolean literal \lstinline|\\false|.
	\item If a function with a \lstinline|[[noexcept]]| attribute has a behavior
	that contains no \lstinline|throws| clause, the default \lstinline|throws|
	clause is \lstinline|throws \\false;| (rather than the usual \lstinline|throws \true;|).
\end{itemize}

An example of this is shown here:
\begin{listing-nonumber}
//@ throws  d == 0;  // ILLEGAL in conjunction with noexcept
int m(int d) noexcept {}
\end{listing-nonumber}

\ifImpl{The syntactic checks are not yet performed.}
